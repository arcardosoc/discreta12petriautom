\hyperlink{ex12_8c}{ex12.\+c} trata-\/se de uma simulação de uma rede de Petri, se utilizando dos recursos de dados abstratos e de threads. A entrada de dados será feita pela leitura dos valores nos arquivos entradas-\/petri.\+txt e os valores são armazena dos em variáveis, sendo os cinco primeiros valores representativos de quantidades (lugares, transições...), os demais não são variáveis únicas sendo necessário a utilização do sistema de lista para armazena-\/las. O paralelismo foi implementado para simular a aleatoriedade do programa, além de melhorar o processamento desta. Para cada thread haverá uma transição com listas de arcos que partem ou entram nela, denominadas listas de arco entram e de arco saem. Com base nessas listas arco é que a simulação funciona, retirando tokens, ativando transições e adicionando tokens em novos lugares ou no mesmo lugar. A rede de Petri é desenha para melhor visualização e para isso é utilizado a biblioteca allegro.\+h, na qual 3 funções derivam para desenhar tal rede, sendo elas (desenha\+\_\+estados, desenha\+\_\+transicao e desenha\+\_\+arcos).

\begin{DoxyVerb}Alunos: Arthur Carvalho de Albuquerque Cardoso
        Mateus Lenier Rezende
Professor: Dr. Ruben Carlo Benante.
Curso: Engenharia de Controle e Automação.\end{DoxyVerb}
 